In this exercise sheet we will look at systems of linear equations and how to
solve them. Consider the system of equations
\begin{align*}
    x + y + 2z &= 3\\
    x - 2y + 3z &= -5\\
    2x + 2y + 2z &= 4
    \label{system1}
\end{align*}
There are things we can do to solve this system of equations. We can multiply
any equation by a non-zero number. We can add and subtract equations to make
new equations. We can solve for one variable in one of the equations and use
that to eliminate that variable from all of the other equations. The technique
we are going to use here is called Gaussian elimination and consists in a
combination of adding and multiplying equations to transform the system into a
form where it is easily solved.

First note that we can write this system as a single \emph{matrix equation}
$\mathrm{Mx} = \mathrm{y}$ which looks like
\[
    \begin{pmatrix}
        1 & 1 & 2 \\
        1 & -2 & 3 \\
        2 & 2 & 2
    \end{pmatrix}
    \begin{pmatrix}
        x \\ y \\ z
    \end{pmatrix}
       =
    \begin{pmatrix}
        3 \\ -5 \\ 4
    \end{pmatrix}
\]
In turn we can write this more compactly as an \emph{augmented matrix}
\[
    \begin{amatrix}{3}
        1 & 1 & 2 & 3 \\
        1 & -2 & 3 & -5 \\
        2 & 2 & 2 & 4
    \end{amatrix}
\]
We are now in a position to solve this using Gaussian elimination. Step 1 is
to replace the second row with the result of subracting the first from the
second. Our notation for this is $r_2 \to r_2 - r_1$. The next step is $r_3
\to r_3 - 2r_1$. At this point the augmented matrix looks like
\[
    \begin{amatrix}{3}
        1 & 1 & 2 & 3 \\
        0 & -3 & 1 & -8 \\
        0 & 0 & -2 & -2
    \end{amatrix}
\]
Our aim in these row operations is to reduce the matrix to this form in which
we have all zeros below the main diagonal. Normally we do this iteratively by
first using the top row to make all zeros below the top row in the first
column. Then we use the second row to make all zeros below the second row in
the second column and so on. In this particular case it turns out that the
second column is already sorted so we don't need any more row operations.
If we rewrite this as separate equations we find
\begin{align*}
    x +y + 2z &= 3\\
     -3y + z  &= -8\\
          -2z &= -2
    \label{system1}
\end{align*}
which can be solved by back-substitution. How that works is that the last
equations gives us the value of $z$ i.e. $z=1$. We can then substitute this
into the second equation and find that $y=3$. Now that we have $y$ and $z$ we
can substitute both into the first equation to find that $x=-2$.
\clearpage
