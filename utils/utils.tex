% Include the headers needed by the questions here
\usepackage{probsoln}
\usepackage{graphicx}
\usepackage{amsmath}
\usepackage{amssymb}
\usepackage{../utils/shortlst}
\usepackage{color}

%
% change list-making
%
%\renewcommand{\theenumi}{\alph{enumi}}
%\def\labelenumi{(\theenumi)}
%\renewcommand{\theenumii}{\roman{enumii}}
%\def\labelenumii{(\theenumii)}

%
% The probsoln package already has the \newproblem and \newproblem* commands
% similar to \problem below. Apart from not supporting as many options, the
% main difference is that this \problem command does not show the original
% problem at the same time as the answer.
%
% I prefer this since it means that I can put the questions and answers on the
% same sheet and the students won't be too tempted to spend all their time
% reading from the solutions part (as they would if it had both the questions
% and the answers).
%
\newcommand{\problem}[4]{
  \begin{defproblem}{#1}
    #2    % Header for problem and solution (use for \newcommand etc)
    \begin{onlyproblem}
      #3  % Problem
    \end{onlyproblem}
    \begin{onlysolution}
      #4  % Solution
    \end{onlysolution}
  \end{defproblem}
}

\newcommand{\Q}[1]{
  %\begin{onlyproblem}
    #1
  %\end{onlyproblem}
}

\newcommand{\A}[1]{
  \begin{onlysolution}
  \textcolor{blue}{
    #1
  }
  \end{onlysolution}
}

\newcommand{\An}[1]{
  \begin{onlysolution}

  \fbox{\parbox{0.95\linewidth}{\A{#1}}}
  \end{onlysolution}
}

\newcommand{\E}[1]{} % no-op - use to disable answers altogether

\newcommand{\QA}[2]{\Q{#1}\A{#2}}

\newcommand{\multiproblem}[2]{
  \begin{defproblem}{#1}
    #2
  \end{defproblem}
}

% Custom figure command
\newcommand{\qfigure}[2]{
  \begin{center}
  \includegraphics{#1}\newline
  {\it
    #2
  }
  \end{center}
}

% Trigonometric
\newcommand{\cosec}{\mathrm{cosec}\,}
\newcommand{\degrees}{\ensuremath{^\circ}}

% Vectors
\newcommand{\ei}{\ensuremath{\mathbf{i}}}
\newcommand{\ej}{\ensuremath{\mathbf{j}}}
\newcommand{\ek}{\ensuremath{\mathbf{k}}}

% Calculus
\newcommand{\dd}{\mathrm{d}}
\newcommand{\dx}[1]{\,\dd #1}
\newcommand{\dydx}[2]{\frac{\dd#1}{\dd#2}}
\newcommand{\dydxsq}[2]{\frac{\dd^2#1}{\dd#2^2}}
\newcommand{\ddx} [2]{\frac{\dd}{\dd#1}\left(#2\right)}
\newcommand\at[2]{\left.#1\right|_{#2}}

% Units
\newcommand{\newtons}{\ensuremath{\,\mathrm{N}}}
\newcommand{\knewtons}{\ensuremath{\,\mathrm{kN}}}
\newcommand{\kN}{\ensuremath{\,\mathrm{kN}}}
\newcommand{\kNm}{\ensuremath{\,\mathrm{kNm}}}
\newcommand{\cm}{\ensuremath{\,\mathrm{cm}}}
\newcommand{\cmsq}{\ensuremath{\,\mathrm{cm}^2}}
\newcommand{\cmcu}{\ensuremath{\,\mathrm{cm}^3}}
\newcommand{\metres}{\ensuremath{\,\mathrm{m}}}
\newcommand{\mps}{\ensuremath{\,\mathrm{ms}^{-1}}}
\newcommand{\mpssq}{\ensuremath{\,\mathrm{ms}^{-2}}}

\newcommand{\real}{\operatorname{Re}}
\newcommand{\imag}{\operatorname{Im}}

\newcommand{\Mod}[1]{\ (\text{mod}\ #1)}


\newenvironment{amatrix}[1]{%
  \left(\begin{array}{@{}*{#1}{c}|c@{}}
}{%
  \end{array}\right)
}
