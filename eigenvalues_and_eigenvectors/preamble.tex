In previous worksheets we have discussed how to solve systems of linear equations of the form
\begin{equation}\label{lin}
  {\bf A} { \bf x}= {\bf b}
\end{equation}
where ${\bf A}$ is a known square $n\times n$ matrix, ${\bf b}$ is a known
$n\times 1$ vector and ${\bf x}$ is an $n\times 1$ vector of \emph{unknowns}.
The technique we have used for this is called Gaussian elimination.
First, the augmented matrix $[  {\bf A}|{ \bf b} ]$ can be constructed, and then row operations can be used to put the matrix into upper triangular (or echelon) form. It is then possible to solve each component of ${\bf x}$ using back substitution starting from the bottom row.

This system has a unique solution if and only if $\text{det}( {\bf A})$ is non-zero. If $\text{det}( {\bf A})=0$, then there are either no solutions or infinitely many. In particular, if $\text{Rank}( {\bf A}) < \text{Rank}( [{\bf A}|{\bf b}])$ there is no solution. If $\text{Rank}( {\bf A}) = \text{Rank}([ {\bf A}|{\bf b}])$ there is an infinite family of solutions.

The system \eqref{lin} is called {\em homogeneous} in the case where ${\bf
b}={\bf 0}$. In this case equation~\eqref{lin} becomes instead
\begin{equation}
    {\bf A x} = {\bf 0}
    \label{eq:homo}
\end{equation}
