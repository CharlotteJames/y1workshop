This workshop will develop your ability to carry out a number of different proof
techniques. Specifically we want to look at proof by \emph{induction},
\emph{contrapositive} and \emph{contradiction}. We first summarise the basic
idea behind each of these methods.

Proof by contradiction is an indirect proof technique. This technique applies in
any situation where we might want to prove something. Simply we want to prove
a proposition $A$ and we do so by showing that $\neg A$ (not $A$) leads to a
contradiction. It follows then that $\neg A$ is false and therefore that $A$
is true. The method in steps is
\begin{enumerate}
  \item Assume $\neg A$
  \item Use that to prove a contradiction (or an obviously false statement)
  \item Conclude that $A$ is true.
\end{enumerate}

Proof by contrapositive is an indirect way of proving a statement of the form
$A\implies B$ by instead proving that $\neg B \implies \neg A$. In words we
want to prove that \emph{if A is true, then B is true}. We prove this
indirectly by proving that \emph{if B is false, then A is false}.
These two statements are logically equivalent but usually it is easier to
phrase a proof for one than the other. When we prove that $A\implies B$
indirectly like this the technique is known as \emph{proof by contrapositive}.
The method is summarised as
\begin{enumerate}
  \item Assume $\neg B$ (that $B$ is false)
  \item Use that assumption to prove $\neg A$
  \item Conclude that $A\implies B$.
\end{enumerate}

Proof by induction applies to a situation where we have a statement
parametrised by an integer which we denote as $P(n)$. For example we might
have
\begin{equation} \label{eq:ex1}
 P(n):=\quad 1+2+3+\dots+n=\frac{n(n+1)}{2}\quad\forall ~n\in\mathbb{N}.
\end{equation}
The statement will typically hold for any natural number $n=0,1,2,3\dots$. The
method works by first proving $P(0)$ and then proving that $P(n)\implies
P(n+1)$ for \emph{all} $n$. It follows that $P(n)$ is true for all $n$. Note
that in the inductive step we will typically ``assume'' that $P(n)$ is true to
show that it implies $P(n+1)$ - this is known as the \emph{inductive
hypothesis}. Please understand the difference between this assumption and
assuming the result that we are trying to prove which is that $P(n)$ is true
for \emph{all} $n$. The method is summarised as
\begin{enumerate}
  \item Prove the base step: $P(0)$
  \item Prove the inductive step: $P(n) \implies P(n+1) \quad\forall ~n\in \mathbb{N}$
    \begin{enumerate}
      \item Assume $P(n)$ - the inductive hypothesis
      \item Use that to prove $P(n+1)$.
    \end{enumerate}
    (Alternatively the inductive step can be proved by
    contradiction/contrapositive)
  \item Conclude that: $P(n) \quad\forall ~n \in \mathbb{N}$.
\end{enumerate}

\clearpage
