\multiproblem{7}{
  Use proof by contradiction to prove that
  \begin{enumerate}
    \item There are no integer solutions to the equation $x^3+x=1$.
    \item There are no integers $a$ and $b$ for which $6a+9b=1$.
    \item $\sqrt{3}$ is irrational.
    \item If $a$ is rational and $ab$ is irrational, then $b$ is irrational.
    \item For all $n \in \mathbb{Z}$ if $n$ is odd, then $n^2$ is odd.
    \item For all $n \in \mathbb{Z}$ if $n^2$ is odd, then $n$ is odd.
  \end{enumerate}
}

\multiproblem{8}{Use proof by contrapositive to prove that
  \begin{enumerate}
    \item For $x\in\mathbb{R}$ if $x^2 + 5x < 0$ then $x < 0$.
    \item For $x,y\in\mathbb{Z}$ if $xy$ is even, then $x$ is even or $y$ is even.
    \item For $n\in\mathbb{Z}$ if $3\nmid n^2$, then $3 \nmid n$.
    \item For $a,b,c\in\mathbb{Z}$ if $a\nmid bc$ then $a\nmid b$.
    \item For $a,b\in\mathbb{Z}$ if $ab$ and $a+b$ are both even, then $a$ is
      even and $b$ is even.
  \end{enumerate}
}

\multiproblem{1}{Prove that the following statements hold $\forall~n\in\mathbb{N}$ by induction
\begin{enumerate}
\item \[ P(n):=\quad \sum_{i=1}^{n} i = \frac{1}{2} n (n+1) \]
\item \[ P(n):=\quad \sum_{i=1}^{n}(2i-1)=n^2. \]
\item \[ P(n):=\quad \sum_{i=0}^{n} 2^i = 2^{n+1}-1. \]
\item \[ P(n):=\quad \sum_{i=0}^{n}r^{i}=\frac{1-r^{n+1}}{1-r}, \quad r\neq 1.\]
\end{enumerate}
}

\multiproblem{2}{By induction prove that
\begin{enumerate}
 \item $(1+p)^{n} \geq 1+np$, where $p\ge 0$, $n\in \mathbb{N}$.
 \item $n^{3}+2n$ is divisible by 3, for $n \in \mathbb{N}$.
 \item $6^{n}-1$ is divisible by 5, for $n \in \mathbb{N}$.
 \item the number $2^{4n-1}$ ends with an 8, for $n \in \mathbb{N}$.
 \item $n!>2^{n}$ for $n\in\mathbb{N}$ and $n\geq 4$.
 \item Use induction and the product rule of calculus to show prove that
   $\frac{\mathrm{d}}{\mathrm{d}x}x^{n}=nx^{n-1}$, for $n \in
   \mathbb{N}$.
 \item Prove that $\frac{\mathrm{d}^{n}}{\mathrm{d}x^{n}}x^{n}=n!$, for $n \in
   \mathbb{N}$.
 \item Prove that $\int_{0}^{\infty}x^{n}\mathrm{e}^{-x}\mathrm{d}x=n!$, for
   $n \in \mathbb{N}$. (Hint: Use integration by parts).
\end{enumerate}
}

\multiproblem{6}{The Binomial Theorem states that
\begin{equation}
 (a+b)^{n}=\sum_{r=0}^{n} \binom{n}{r} a^{r} b^{n-r}
 \label{eq:bin}
\end{equation}
where $a,b \in \mathbb{R}$ and $n\in \mathbb{N}$. The binomial coefficient is defined
\[
 \binom{n}{r}=\frac{n!}{r!(n-r)!}.
\]
\begin{enumerate}
 \item Show that
 \[
  \binom{n}{r-1}+\binom{n}{r}=\binom{n+1}{r}.
 \]
 \item Establish that the Binomial Theorem holds where $n=1$ (the Base Step).
 \item Show that if we assume \eqref{eq:bin} (i.e. that the Binomial Theorem holds for $n$) then
 \[
  (a+b)^{n+1}=(a+b)(a+b)^{n}=\sum_{r=1}^{n+1}\binom{n}{r-1}a^{r}b^{n+1-r}+\sum_{r=0}^{n}\binom{n}{r}a^{r}b^{n+1-r}.
 \]
(Hint: the trick is to make a substitution of $s=r+1$ in the first summation
term after expanding).
\item Prove the Binomial Theorem by induction (Hint: You have already
  demonstrated the Base Step. To complete the Inductive Step, work from the
  result you have found in (c). It will be helpful to use the result regarding
  binomial coefficients that you found in (a)).
\end{enumerate}
}
