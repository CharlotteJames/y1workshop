In this worksheet we consider planar statics problems, and how to solve the linear systems of equations that arise when solving them.

From Newton's laws of motion, for the static equilibrium of a body both translational forces and moments must balance. Specifically, the vector sum of the external forces must equal zero:
\begin{equation} \label{eq:trans}
 \sum_{i} {\bf F}_{i}= \bf{0}
\end{equation}
This is called the {\em external force balance}. Also, the sum of moments, $\bf{M}$, about an arbitrary point ${\bf r}$ of all forces is zero:
\begin{equation} \label{eq:mom}
 \sum_{i} {\bf M}_{i}= \sum_{i} ({\bf r}_{i}-{\bf r}) \times {\bf F}_{i}= 0.
\end{equation}
This is called the {\em external torque balance}. It can be proven that there is no additional linearly independent balance solely involving the external forces.

Now for planar statics problems, \eqref{eq:trans} becomes
\[
 {\bf F}_{i}=\begin{pmatrix}
         F_{i,x} \\ F_{i,y} \\ 0
        \end{pmatrix}
\]
so it follows that the external force balance yields two equations, $\sum F_{i,x} =0$ and $\sum F_{i,y}=0$. For such problems the magnitude of the moment of a given force can be found by simply multiplying the force by the perpendicular distance through which it levers. The external torque balance \eqref{eq:mom} yields a single equation, so there are three equations in total for static equilibrium.

For a given statics problem with more unknowns than equations, the system of simultaneous equations does not have a unique solution and the problem is said to be {\em statically indeterminate}.

If one body is {\em resting} on another, the reaction force $N$ must be positive.

If friction acts a point of contact, the friction condition $F\leq \mu N$ must always be satisfied.

The force in a spring of stiffness $k$ with extension $x$ from its natural length $L$ is given by Hooke's law, $F=kx$.
