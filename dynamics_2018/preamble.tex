
The position of a particle is generally expressed as the vector ${\bf r}(t)$. Then the velocity is ${\bf v}(t)=\dot{\bf r}(t)$ and the acceleration is ${\bf a}(t)=\ddot{\bf r}(t)$.

  
The {\em work} done by a force ${\bf F}$ action on a mass moving along path $C_{12}$ between $P_{1}$ and $P_{2}$ is:  
$$W=\int_{C_{12}}{\bf F} \cdot {d {\bf r}} = \int_{C_{12}}{\bf F}\cdot {\bf v} dt.$$
%
The {\em kinetic energy} of a particle is the work done by a force to accelerate it from rest to speed $v$: $$T = \frac{1}{2} m v^2.$$
%
The Work-Energy Theorem states that the work done by the {\em resultant force} is equal to the change of kinetic energy of the mass: 
$$W=\int_C {\bf F} \cdot {d {\bf r}}=T(t_{1})-T(t_{0}) = \Delta T.$$

The work done on the mass by the resultant force is equal to the sum of the work done by each individual force: 
$$ \Delta T = W = \sum_i W_i.$$

A force is {\em conservative} if its work depends only on the initial and final positions of the mass, not on the path taken:
$$W_{C}(P_1,P_2) = \int_{C_{12}}{\bf F} \cdot {d {\bf r}} = V(P_1) - V(P_2) = -\Delta V.$$
Conservative forces are time independent or constant.
  
Scalar function $V$ is the potential energy. It is defined only for {\em conservative} forces and it is the opposite of the work done by the force on the mass between points $P_1$ and $P_2$:
$$V(P_1,P_2) = V(P_2) - V(P_1) = - W(P_1,P_2) = - \int_{P_1}^{P_2} {\bf F}\cdot{d{\bf r}}.$$ 
Examples of potential energy are gravitational potential energy, $V = mgh$ and elastic potential energy, $V = \frac{1}{2}kx^2$.

Forces of {\em constraint} act to constrain the mass within a boundary; they ensure there is no velocity in the direction of the constraint. For example, a mass on a slope with gravity is constrained by the normal reaction to remain on the surface of the slope.

The total work done on a mass is equal to the sum of the work done by conservative, non-conservative and constraint forces:
$$ W = W_{conservative} + W_{non-conservative} + W_{constraint}.$$

If the only forces acting on a mass are conservative ones, combining the above expression with the work-energy theorem we have:
$$ \Delta T =  W_{conservative} = -\Delta V.$$
The change in kinetic energy $\Delta T$ is equal to minus the change in potential energy, $\Delta V$. As a consequence, the total energy of the system is conserved:
$$ E = T + V$$
$$\Delta E = \Delta T + \Delta V = 0. $$





