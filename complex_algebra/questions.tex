%
%

\multiproblem{complex}{
  \begin{enumerate}
      \item Complex numbers: A complex number $z$ is any number that can be
          represented in the form $z = x+j\,y$ where $x$ and $y$ are real
          numbers ($x,y\in\mathbb{R}$) and $j^2 = -1$. The set of complex
          numbers is denoted $\mathbb{C}$ and we say that $z\in\mathbb{C}$.
      \item Real and imaginary parts: Given $z = x+j\,y$ with $x,y\in\mathbb{R}$
          we define the real and imaginary parts of $z$ as $\real(z)=x$ and
          $\imag(z)=y$.

          Given $x=2$ and $y=1+j$ if $z=x+j\,y$, what is $\real(z)$? Why
          isn't it equal to $x$?
      \item Complex conjugate: The complex conjugate of a complex number
          $z=x+j\,y$ with $x,y\in\mathbb{R}$ is denoted $z^*$ and given by $z^*=x-j\,y$.

          Given $x=2$ and $y=1+j$ as above, is $z^*$ equal to $x-j\,y$?
      \item Distributivity of real parts (example): given two complex numbers
          $z_1$ and $z_2$ show that if $z_3=z_1+z_2$ then $\real(z_3)$ =
          $\real(z_1) + \real(z_2)$.  More generally we have that
          \[
              \real(z_1+z_2) = \real(z_1) + \real(z_2)
          \]
          We can prove this by noting that if $z_1$ is a complex number then
          (by the definition of complex numbers above) there must be real
          numbers $x_1$ and $y_1$ such that $z_1 = x_1 + j\,y_1$ and likewise
          $z_2 = x_2+j\,y_2$. We have then that
          \[
              \real(z_1+z_2) = \real((x_1 + j\,y_1) + (x_2 + j\,y_2)
                             = \real((x_1 + x_2) + j\,(y_1 + y_2)).
          \]
          Now since $x_1$ and $x_2$ are real we know that $x_1 + x_2$ is real
          and likewise for $y_1+y_2$. It follows from the definition of the
          real part that $x_1+x_2$ will be the real part so that
          \[
              \real(z_1+z_2) = x_1 + x_2 = \real(z_1) + \real(z_2)
          \]
          as required.
      \item Exercises: Assuming $z,z_1,z_2\in\mathbb{C}$ and $r\in\mathbb{R}$
          which of the following statements is true? If a statement is true
          then prove it. Otherwise disprove it. If it is sometimes true then
          under what conditions is it true or false? In all cases you must
          only use the definitions above in these problems.
          \begin{enumerate}
              \item $\imag(z_1 + z_2) = \imag(z_1) + \imag(z_2)$

        \A{True. For the whole question let $z_i = x_i +jy_i$, for $i=1,2$, such that $\real(z_i)=x_i$ and $\imag(z_i)=y_i$. So we have
\begin{equation*}
\begin{split}
  \imag(z_1+z_2) &= \imag(x_1+jy_1+x_2+jy_2) \\
      &= y_1 + y_2 \\
      &= \imag(z_1) + \imag(z_2).
\end{split}
\end{equation*}}

              \item $\real(jz) = - \imag(z)$

        \A{True. Splitting into Cartesian form again we have,
\begin{equation*}
\begin{split}
    \real(jz) &= \real(j(x+jy)) \\
        &= \real(jx-y) \\
        &= -y \\
        &= -\imag(z).
\end{split}
\end{equation*}}

              \item $\imag(jz) = - \real(z)$

        \A{False. In fact we find that,
\begin{equation*}
\begin{split}
    \imag(jz) &= \imag(j(x+jy)) \\
        &= \imag(jx-y) \\
        &= x \\
        &= \real(z).
\end{split}
\end{equation*}}

              \item $\real(rz) = r\real(z)$

        \A{True. Remembering $r\in \mathbb{R}$,
\begin{equation*}
\begin{split}
    \real(rz) &= \real(r(x+jy)) \\
        &= rx \\
        &= r\real(z)
\end{split}
\end{equation*}}

              \item $\real(z_1 z_2) = \real(z_1) \real(z_2)$

        \A{False.
\begin{equation*}
\begin{split}
    \real(z_1z_2) &= \real((x_1+jy_1)(x_2+jy_2)) \\
        &= \real(x_1x_2-y_1y_2+j(x_1y_2+x_2y_1)) \\
        &= x_1x_2-y_1y_2 \\
        &= \real(z_1)\real(z_2)-\imag(z_1)\imag(z_2).
\end{split}
\end{equation*}
So we find an extra real term in the expansion of the brackets, coming from the product of the imaginary components.}

              \item $z + z^* = 2\real(z)$

        \A{True. The complex conjugate of $z$ is $z^*$, which we define in Cartesian form as $z^*=x-jy$. Note that $\real(z^*)=x=\real(z)$, and $\imag(z^*)=-y=-\imag(z)$. So we have,
\begin{equation*}
\begin{split}
    z+z* &= (x+jy) + (x-jy) \\
        &= 2x \\
        &= 2\real(z).
\end{split}
\end{equation*}}

              \item $z - z^* = 2\imag(z)$

        \A{False.
\begin{equation*}
\begin{split}
    z-z^* &= (x+jy) - (x-jy) \\
        &=  2jy\\
        &= 2j\imag(z).
\end{split}
\end{equation*}
So we were missing a factor of $j$.}

              \item $(z_1 + z_2)^* = z_1^* + z_2^*$

        \A{True.
\begin{equation*}
\begin{split}
    (z_1+z_2)^* &= ((x_1+jy_1) + (x_2+jy_2))^* \\
        &=  ((x_1+x_2) + j(y_1+y_2))^*\\
        &= ((x_1+x_2) - j(y_1+y_2)) \\
        &= (x_1-jy_1) + (x_2 - jy_2) \\
        &= z_1^*+z_2^*.
\end{split}
\end{equation*}}

              \item $(z_1 z_2)^* = z_1^* z_2^*$

        \A{True.
\begin{equation*}
\begin{split}
    (z_1z_2)^* &= ((x_1+jy_1)(x_2+jy_2))^* \\
        &=  ((x_1x_2 - y_1y_2) + j(x_1y_2+x_2y_1))^*\\
        &= ((x_1x_2 - y_1y_2) - j(x_1y_2+x_2y_1)) \\
        &= (x_1-jy_1)(x_2-jy_2) \\
        &= z_1^*z_2^*
\end{split}
\end{equation*}}

          \end{enumerate}
  \end{enumerate}
}

\multiproblem{Arithmetic}{
  \begin{enumerate}
      \item Modulus: The modulus of a complex number $z$ is denoted $|z|$
          and is defined by $|z|=\sqrt{zz^*}$.
          \begin{enumerate}

              \item Show that $|z|$ is real for any complex number
          \A{Define $z=x+jy$, where $x,y \in \mathbb{R}$, so $z^*=x-jy$. \\
               Using $|z|=\sqrt{zz^*}$, $|z| =\sqrt{(x+jy)(x-jy)}$, \\
               which expands to $|z|= \sqrt{x^2+y^2}$. Given that $x,y \in \mathbb{R}$,
        $|z|$ is always real.}
              \item Show that $|z|\ge 0$ for any complex number
          \A{From the above expression, $\sqrt{x^2+y^2} \geq 0$, as $x$ and $y$
        are real, so $x^2$ and $y^2$ will be positive, therefore $|z|\geq 0$.}
              \item Show that $|z| = |z^*|$.
        \A{Using the definition $|z|=\sqrt{zz^*}$, $|{z^*}|=\sqrt{z^*(z^*)^*}$,
        and $(z^*)^*=z$, therefore $|{z^*}|=\sqrt{z^*(z^*)^*}=\sqrt{zz^*}$}
              \item Show that if $z=x+j\,y$ with $x,y\in\mathbb{R}$ then
                  $|z|^2 = x^2 + y^2$.
        \A{$z=x+jy$ and $z^*=x-jy$ so $|z|^2=(\sqrt{(x+jy)(x-jy)})^2$. Expanding
        this leads to $|z|^2=x^2+y^2$.}
        \end{enumerate}

              \item Which complex numbers satisfy $|z|^2 = z^2$?
        \A{$z^2=(x+jy)^2=x^2-y^2+2xyj$. From the previous question, $|z|^2=x^2+y^2$,
        so setting the right-hand side of these equations equal to each other gives
        $x^2-y^2+2xyj=x^2+y^2$, which cancels down to $0=y(2xj-2y)$.}
      \item Inverse: Prove that for any non-zero complex number $z$ there
          exists a multiplicative inverse $z^{-1}$ such that $zz^{-1} = 1$.
          Find a formula for $z^{-1}$ in terms of $z$.
    \A{We know that $z \neq 0$, so $|z| \neq 0$. If we then assume that
    $z^{-1}=\frac{z^*}{|z|^2}$, it can be shown that
    $zz^{-1}=z.\frac{z^*}{|z|^2}=\frac{{|z|^2}}{{|z|^2}}=1$.}
      \item Divisibility: We say that a complex number $z_2$ divides another
          complex number $z_1$ if there exists a complex number $z_3$ such
          that $z_1 = z_2 z_3$. In this instance we would write that $z_3 =
          \frac{z_1}{z_2}$. Show that every \emph{non-zero} complex number
          divides every complex number and find a formula for $z_3$ in terms
          of $z_1$ and $z_2$.
    \A{Given that $z_2 \neq 0$, there exists $z_2^{-1}$. We can therefore say that
    $z_3=z_1z_2^{-1}$. It can then be shown that $z_2z_3=z_2z_1z_2^{-1}=z_1$.}

      \item Which of the following is true? Prove/disprove as necessary.
          \begin{enumerate}
              \item $|z_1 z_2| = |z_1||z_2|$
        \A{True: taking $z_{1}=x_{1}+jy_{1}$ and $z_{2}=x_{2}+jy_{2}$,
        $|{z_{1}z_{2}}|=|{(x_{1}+jy_{1})(x_{2}+jy_{2})}|$ \\
             Expanding this out makes $|{x_{1}x_{2}+x_{2}jy_{1}+jy_{2}x_{1}-y_{1}y_{2}}|=
        |{x_{1}x_{2}-y_{1}y_{2}+j(x_{2}y_{1}+y_{2}x_{1})}|$.
        Using $|{z}|=\sqrt{zz^*}, |{x_{1}x_{2}-y_{1}y_{2}+j(x_{2}y_{1}+y_{2}x_{1})}|$
        $= \\ \sqrt{(x_{1}x_{2}-y_{1}y_{2}+j(x_{2}y_{1}+y_{2}x_{1}))(x_{1}x_{2}-y_{1}y_{2}-j(x_{2}y_{1}+y_{2}x_{1}))} $
             which expands to $|{z_{1}z_{2}}|$
        $=\sqrt{(x_{1}x_{2}-y_{1}y_{2})^2+(x_{1}y_{2}-y_{1}x_{2})^2}$ .\\
             This can be rearranged into the form $|{z_{1}z_{2}}|$
        $=\sqrt{(x_{1}^{2}+y_{1}^2)(x_{2}^{2}+y_{2}^2)}$, \\
             which then becomes $|{z_{1}z_{2}}|=\sqrt{x_{1}^{2}+y_{1}^2}.\sqrt{x_{2}^{2}+y_{2}^2}=|{z_{1}}||{z_{2}}|$.}
        \item $|z_1 + z_2| = |z_1| + |z_2|$
        \A{False: taking $z_{1}=x_{1}+jy_{1}$ and $z_{2}=x_{2}+jy_{2}$, $|z_1+z_2|$
        $=|{x_1+x_2+j(y_1+y_2)}|=\sqrt{(x_1+x_2)^2+(y_1+y_2)^2}$ \\
        So $|{z_1}|+|z_2|=\sqrt{x_1^2+y_1^2}+\sqrt{x_2^2+y_2^2} \neq |z_1+z_2|$}
              \item $|z^{-1}| = |z|$
        \A{False: using $z^{-1}=\frac{z^*}{|z|^2}$
        and substituting in $z=x+jy$, $z^{-1}=\frac{x-jy}{x^2+y^2}$ \\
             Therefore, $|z^{-1}|=\sqrt{\frac{x-jy}{x^2+y^2}.\frac{x+jy}{x^2+y^2}}$. \\
             This simplifies to $z^{-1}=\frac{\sqrt{x^2+y^2}}{x^2+y^2}$,
        which is not equal to $|z|=\sqrt{x^2+y^2}$}
              \item $|\frac{z_1}{z_2}| = \frac{|z_1|}{|z_2|}$
        \A{True: $|{\frac{z_1}{z_2}}|=|{\frac{x_1+jy_1}{x_2+jy_2}}|$=$
        |{\frac{(x_1+jy_1)(x_2-jy_2)}{x_2^2+y_2^2}}|$
        $=|{\frac{(x_1x_2+y_1y_2)+j(y_1x_2-y_2x_1)}{x_2^2+y_2^2}}|=$ \\
        $\sqrt{\frac{(x_1x_2+y_1y_2)^2+(y_1x_2-y_2x_1)^2}{{x_2^2+y_2^2}^2}}$. \\
        This can be expanded out and cancelled down to form
        $|{\frac{z_1}{z_2}}|=\sqrt{\frac{(x_1^2+y_1^2)(x_2^2+y_2^2)}{(x_2^2+y_2^2)^2}}=$
        $\sqrt{\frac{(x_1^2+y_1^2)}{(x_2^2+y_2^2)}}=\frac{{|z_1|}}{|z_2|}$}

              \item $\left(\frac{z_1}{z_2}\right)^* = \frac{z_1^*}{z_2^*}$
        \A{True: using $(\frac{z_{1}}{z_{2}})^* = (\frac{x_{1}+jy_{1}}{x_{2}+jy_{2}})^*$
        and then rationalising to get $(\frac{z_{1}}{z_{2}})^*$=
        $(\frac{(x_{1}+jy_{1})(x_{2}-jy_{2})}{x_{2}^2+jy_{2}^2})^*$ \\
             this can be rearranged to the form $(\frac{z_{1}}{z_{2}})^*$=
        $(\frac{x_{1}x_{2}+y_{1}y_{2}+i(y_{1}x_{2}-x_{1}y_{2})}{x_{2}^2+jy_{2}^2})^*$.
        Applying the conjugate gives $(\frac{z_{1}}{z_{2}})^*$=$(\frac{x_{1}x_{2}+y_{1}y_{2}-i(y_{1}x_{2}-x_{1}y_{2})}{x_{2}^2+jy_{2}^2})$, \\
             which can be rearranged to $(\frac{x_{1}x_{2}+y_{1}y_{2}-i(y_{1}x_{2}-x_{1}y_{2})}{x_{2}^2+jy_{2}^2})$=
        $(\frac{(x_{1}-jy_{1})(x_{2}+jy_{2})}{x_{2}^2+jy_{2}^2})$=$\frac{x_{1}-jy_{1}}{x_{2}-jy_{2}}$=$\frac{z_{1}^*}{z_{2}^*}$}

          \end{enumerate}
      \item Problem: Given $a,b,z\in\mathbb{C}$ and $|z|=1$ show that
          $\left|\frac{az + b}{b^*z + a^*}\right| = 1$
    \A{From the relationship $|{z}|^2=zz^*$, $|{\frac{az+b}{b^*z+a^*}}|=\sqrt{\frac{(az+b)(az+b)^*}{(b^*z+a*)(bz^*+a)}}$. \\
     This expands out to $|{\frac{az+b}{b^*z+a^*}}|$=$\sqrt{\frac{aa^*zz^*+bb^*+ba^*z*+b^*az}{bb^*zz^*+a^*bz^*+aa^*+ab^*z}}$ \\
     We also know that $|{z}|=1$, which means that $\sqrt{zz^*}=1$, which leads to $zz^*=1$. This can be substituted into the
     above expression, leading to $|{\frac{az+b}{b^*z+a^*}}|$=$\sqrt{\frac{aa^*+bb^*+ba^*z*+b^*az}{bb^*+a^*bz^*+aa^*+ab^*z}}=1$.}

\end{enumerate}
}

\multiproblem{Polynomials}{
    \begin{enumerate}
        \item
            \begin{enumerate}
                \item Show that $(z^*)^2 = (z^2)^*$.
        \A{Define $z=x+jy$, where $x,y \in \mathbb{R}$, so $z^*=x-jy$. \\
           $(z^*)^2=(x-jy)^2=x^2+y^2-j(2y)=(x^2+y^2+j(2y))^*=((x+jy)^2)^*$}
                \item Show that $(z^*)^n = (z^n)^*$ for positive integer $n$ ($n\in\mathbb{N}$).
        \A{$(z^*)^n=(\frac{|z|^2}{z})^n=((\frac{|z|^2}{z^*})^n)^*=(z^n)^*$
        (bearing in mind $|z|^*=|z|$, as $|z|$ is real for any complex number,
        and the conjugate of a real number is itself).}
                \item Show that $(z^*)^ {-1} = (z^{-1})^*$.
        \A{$(z^*)^{-n}=(\frac{z}{|z|^2})^n=((\frac{z^*}{|z|^2})^n)^*=(z^{-n})^*$}
                \item Show that $(z^*)^ {-n} = (z^{-n})^*$ for $n\in\mathbb{N}$.
        \A{From parts ii and iv, this statement holds, as i demonstrates $(z^*)^n=z^n)^*$
        for all positive integers, and iv demonstrates $(z^*)^n=z^n)^*$ for all negative integers,
        which covers $n \in \mathbb{N}$.}
                \item Show that $(z^*)^n = (z^n)^*$ for integer $n$ ($n\in\mathbb{Z}$).
        \A{$(rz)^*=r^*z^*$ and as $r \in \mathbb{R}$, $r*=r$, so $(rz)^*=rz^*$}
                \item Show that $(rz)^* = rz^*$ for $r\in\mathbb{R}$.
            \end{enumerate}
        \item Given a quadratic polynomial
            \[
                P(z) = a\,z^2 + b\,z + c
            \]
            with $a,b,c \in \mathbb{R}$ show that if $z_1$ is a root then
            $z_1^*$ is also a root.

            Where could the pairs of roots for a quadratic with real
            coefficients lie in the complex plane (draw a diagram)? For
            example consider $P(z)=z^2-2\,z+c$ with $c=0$, $c=1$ or $c=2$.
    \A{Given $P(z)=az^2+bz+c$, substitute in $z_{1}$ to get $P(z_1)=a(z_1)^2+bz_1+c$.
    As $z_1$ is a root, we can set the left hand side to zero, giving $0=a(z_1)^2+bz_1+c$.
    The conjugate of both sides can then be taken, giving $0^*=(a(z_1)^2+bz_1+c)^*$.
    The conjugate of 0 is 0, so $0=(a((z_1)^2)^*+b(z_1)^*+c)$ holds, showing that $z_1$ is also a root.}
        \item Given a polynomial of any order
            \[
                P(z) = a_0 + a_1\,z + a_2\,z^2 + \cdots
            \]
            with $a_i \in \mathbb{R}$ show that if $z_1$ is a root then
            $z_1^*$ is also a root.
    \A{The same explanation applies --- if $z_1$ is a root, then $z_1$
    can be substituted in and $P(z)$ set to zero. The conjugate of both sides
    can be taken (given that $a_i \in \mathbb{R}$, the real coefficients will
    remain the same) and since the conjugate of zero is zero, $z_1^*$ must be a root.}
    \end{enumerate}
}

\multiproblem{Eulers}{
    \begin{enumerate}
        \item We define the exponential of an imaginary number by Euler's
            equation
            \[
                \mathrm{e}^{j\,\theta} = \cos{\theta} + j\,\sin{\theta}
            \]
            where $\theta\in\mathbb{R}$. Show that this definition satisfies
            the expected property
            \[
                \mathrm{e}^{j(\theta_1+\theta_2)} =
                \mathrm{e}^{j\theta_1}\mathrm{e}^{j\theta_2}.
            \]

        \A{Using $\mathrm{e}^{j\theta}=\cos\theta+j\sin\theta$, with $\theta=\theta_1+\theta_2$, we have
\begin{equation*}
\mathrm{e}^{j(\theta_1+\theta_2)} = cos(\theta_1+\theta_2)+j\sin(\theta_1+\theta_2).
\end{equation*}
Expanding using the multiple angle formulae for $\cos$ and $\sin$, we find
\begin{equation*}
\begin{split}
\cos(\theta_1+\theta_2)+j\sin(\theta_1+\theta_2) &= \cos\theta_1\cos\theta_2-\sin\theta_1\sin\theta_2 + j(\sin\theta_1\cos\theta_2+\sin\theta_2\cos\theta_1) \\
    &=(\cos\theta_1+j\sin\theta_1)(\cos\theta_2+j\sin\theta_2) \\
    &=\mathrm{e}^{j\theta_1}\mathrm{e^{j\theta_2}},
\end{split}
\end{equation*}
as required.}

        \item We define more generally the exponential of any complex number
            $z=x+j\,y$ ($x,y\in\mathbb{R}$) via
            \[
                \mathrm{e}^z = \mathrm{e}^x\mathrm{e^{jy}}.
            \]
            Show that this definition satisfies the more general relation
            \[
                \mathrm{e}^{z_1+z_2} = \mathrm{e}^{z_1}\mathrm{e}^{z_2}
            \]
            for any $z_1,z_2\in\mathbb{C}$.

        \A{We can write this relation as $\mathrm{e}^z = \mathrm{e}^{\real(z)}\mathrm{e}^{j\imag(z)}$. Using this and $z=z_1+z_2$ we have,
\begin{equation*}
\begin{split}
    \mathrm{e}^{z_1+z_2} &= \mathrm{e}^{\real(z_1+z_2)}\mathrm{e}^{j\imag(z_1+z_2)} \\
        &= \mathrm{e}^{\real(z_1)}\mathrm{e}^{\real(z_2)}\mathrm{e}^{j\imag(z_1)}\mathrm{e}^{j\imag(z_2)} \\
        &= \mathrm{e}^{\real(z_1)+j\imag(z_1)}\mathrm{e}^{\real(z_2)+j\imag(z_2)} \\
        &= \mathrm{e}^{z_1}\mathrm{e}^{z_2},
\end{split}
\end{equation*}
where we have used $\real(z_1+z_2)=\real(z_1)+\real(z_2)$ and $\imag(z_1+z_2) = \imag(z_1) + \imag(z_2)$.}

        \item Show that any complex number $z$ can be written in the form
            $z=r\mathrm{e}^{j\theta}$ with $r=|z|$ and $\theta\in\mathbb{R}$.

        \A{We know we can write $z=x+jy$, so now choose new coordinates such that $x=r\cos\theta$ and $y=r\sin\theta$, for $\theta\in\mathbb{R}$. This choice becomes clear if we draw any $z$ on an Argand diagram and consider the right-angled triangles formed by the axes and the vector $z$. So we have that $z=r\cos\theta+jr\sin\theta$, taking the modulus,
\begin{equation*}
\begin{split}
    |z| &= |r(\cos\theta+j\sin\theta)| \\
        &= \sqrt{r^2(\cos^2\theta-\sin^2\theta)} \\
        &= \sqrt{r^2} \\
        & = r.
\end{split}
\end{equation*}
Then using Euler's formula, we have that $z = r(\cos\theta+j\sin\theta) = r(\mathrm{e}^{j\theta})$, for $r=|z|$ and $\theta\in\mathbb{R}$.}

        \item Show that if $z=r\mathrm{e}^{j\theta}$ then
            \[
                z^*=\mathrm{e}^{-j\theta} = \cos{\theta}-j\sin{\theta}.
            \]

    \A{Again, starting with Cartesian coordinates, if $z^*=x-jy$, then in polar form:
\begin{equation*}
\begin{split}
z^* &= r(\cos\theta-\sin\theta) \\
    &= r(\cos(-\theta) +j\sin(-\theta)) \\
    &= r\mathrm{e}^{-j\theta},
\end{split}
\end{equation*}
via Euler's equation.}

    \end{enumerate}
}
