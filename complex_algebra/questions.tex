%
%

\multiproblem{complex}{
  \begin{enumerate}
      \item Complex numbers: A complex number $z$ is any number that can be
          represented in the form $z = x+j\,y$ where $x$ and $y$ are real
          numbers ($x,y\in\mathbb{R}$) and $j^2 = -1$. The set of complex
          numbers is denoted $\mathbb{C}$ and we say that $z\in\mathbb{C}$.
      \item Real and imaginary parts: Given $z = x+j\,y$ with $x,y\in\mathbb{R}$
          we define the real and imaginary parts of $z$ as $\real(z)=x$ and
          $\imag(z)=y$.

          Given $x=2$ and $y=1+j$ if $z=x+j\,y$, what is $\real(z)$? Why
          isn't it equal to $x$?
      \item Complex conjugate: The complex conjugate of a complex number
          $z=x+j\,y$ with $x,y\in\mathbb{R}$ is denoted $z^*$ and given by $z^*=x-j\,y$.

          Given $x=2$ and $y=1+j$ as above, is $z^*$ equal to $x-j\,y$?
      \item Distributivity of real parts (example): given two complex numbers
          $z_1$ and $z_2$ show that if $z_3=z_1+z_2$ then $\real(z_3)$ =
          $\real(z_1) + \real(z_2)$.  More generally we have that
          \[
              \real(z_1+z_2) = \real(z_1) + \real(z_2)
          \]
          We can prove this by noting that if $z_1$ is a complex number then
          (by the definition of complex numbers above) there must be real
          numbers $x_1$ and $y_1$ such that $z_1 = x_1 + j\,y_1$ and likewise
          $z_2 = x_2+j\,y_2$. We have then that
          \[
              \real(z_1+z_2) = \real((x_1 + j\,y_1) + (x_2 + j\,y_2)
                             = \real((x_1 + x_2) + j\,(y_1 + y_2)).
          \]
          Now since $x_1$ and $x_2$ are real we know that $x_1 + x_2$ is real
          and likewise for $y_1+y_2$. It follows from the definition of the
          real part that $x_1+x_2$ will be the real part so that
          \[
              \real(z_1+z_2) = x_1 + x_2 = \real(z_1) + \real(z_2)
          \]
          as required.
      \item Exercises: Assuming $z,z_1,z_2\in\mathbb{C}$ and $r\in\mathbb{R}$
          which of the following statements is true? If a statement is true
          then prove it. Otherwise disprove it. If it is sometimes true then
          under what conditions is it true or false? In all cases you must
          only use the definitions above in these problems.
          \begin{enumerate}
              \item $\imag(z_1 + z_2) = \imag(z_1) + \imag(z_2)$
              \item $\real(jz) = - \imag(z)$
              \item $\imag(jz) = - \real(z)$
              \item $\real(rz) = r\real(z)$
              \item $\real(z_1 z_2) = \real(z_1) \real(z_2)$
              \item $z + z^* = 2\real(z)$
              \item $z - z^* = 2\imag(z)$
              \item $(z_1 + z_2)^* = z_1^* + z_2^*$
              \item $(z_1 z_2)^* = z_1^* z_2^*$
          \end{enumerate}
  \end{enumerate}
}

\multiproblem{Arithmetic}{
  \begin{enumerate}
      \item Modulus: The modulus of a complex number $z$ is denoted $|z|$
          and is defined by $|z|=\sqrt{zz^*}$.
          \begin{enumerate}
              \item Show that $|z|$ is real for any complex number
              \item Show that $|z|\ge 0$ for any complex number
              \item Show that $|z| = |z^*|$.
              \item Show that if $z=x+j\,y$ with $x,y\in\mathbb{R}$ then
                  $|z|^2 = x^2 + y^2$.
              \item Which complex numbers satisfy $|z|^2 = z^2$?
          \end{enumerate}
      \item Inverse: Prove that for any non-zero complex number $z$ there
          exists a multiplicative inverse $z^{-1}$ such that $zz^{-1} = 1$.
          Find a formula for $z^{-1}$ in terms of $z$.
      \item Divisibility: We say that a complex number $z_2$ divides another
          complex number $z_1$ if there exists a complex number $z_3$ such
          that $z_1 = z_2 z_3$. In this instance we would write that $z_3 =
          \frac{z_1}{z_2}$. Show that every \emph{non-zero} complex number
          divides every complex number and find a formula for $z_3$ in terms
          of $z_1$ and $z_2$.
      \item Which of the following is true? Prove/disprove as necessary.
          \begin{enumerate}
              \item $|z_1 z_2| = |z_1||z_2|$
              \item $|z_1 + z_2| = |z_1| + |z_2|$
              \item $|z^{-1}| = |z|$
              \item $|\frac{z_1}{z_2}| = \frac{|z_1|}{|z_2|}$
              \item $\left(\frac{z_1}{z_2}\right)^* = \frac{z_1^*}{z_2^*}$
          \end{enumerate}
      \item Problem: Given $a,b,z\in\mathbb{C}$ and $|z|=1$ show that
          $\left|\frac{az + b}{b^*z + a^*}\right| = 1$

  \end{enumerate}
}

\multiproblem{Polynomials}{
    \begin{enumerate}
        \item
            \begin{enumerate}
                \item Show that $(z^*)^2 = (z^2)^*$.
                \item Show that $(z^*)^n = (z^n)^*$ for positive integer $n$ ($n\in\mathbb{N}$).
                \item Show that $(z^*)^ {-1} = (z^{-1})^*$.
                \item Show that $(z^*)^ {-n} = (z^{-n})^*$ for $n\in\mathbb{N}$.
                \item Show that $(z^*)^n = (z^n)^*$ for integer $n$ ($n\in\mathbb{Z}$).
                \item Show that $(rz)^* = rz^*$ for $r\in\mathbb{R}$.
            \end{enumerate}
        \item Given a quadratic polynomial
            \[
                P(z) = a\,z^2 + b\,z + c
            \]
            with $a,b,c \in \mathbb{R}$ show that if $z_1$ is a root then
            $z_1^*$ is also a root.

            Where could the pairs of roots for a quadratic with real
            coefficients lie in the complex plane (draw a diagram)? For
            example consider $P(z)=z^2-2\,z+c$ with $c=0$, $c=1$ or $c=2$.
        \item Given a polynomial of any order
            \[
                P(z) = a_0 + a_1\,z + a_2\,z^2 + \cdots
            \]
            with $a_i \in \mathbb{R}$ show that if $z_1$ is a root then
            $z_1^*$ is also a root.
    \end{enumerate}
}

\multiproblem{Eulers}{
    \begin{enumerate}
        \item We define the exponential of an imaginary number by Euler's
            equation
            \[
                \mathrm{e}^{j\,\theta} = \cos{\theta} + j\,\sin{\theta}
            \]
            where $\theta\in\mathbb{R}$. Show that this definition satisfies
            the expected property
            \[
                \mathrm{e}^{j(\theta_1+\theta_2)} =
                \mathrm{e}^{j\theta_1}\mathrm{e}^{j\theta_2}.
            \]
        \item We define more generally the exponential of any complex number
            $z=x+j\,y$ ($x,y\in\mathbb{R}$) via
            \[
                \mathrm{e}^z = \mathrm{e}^x\mathrm{e^{jy}}.
            \]
            Show that this definition satisfies the more general relation
            \[
                \mathrm{e}^{z_1+z_2} = \mathrm{e}^{z_1}\mathrm{e}^{z_2}
            \]
            for any $z_1,z_2\in\mathbb{C}$.
        \item Show that any complex number $z$ can be written in the form
            $z=r\mathrm{e}^{j\theta}$ with $r=|z|$ and $\theta\in\mathbb{R}$.
        \item Show that if $z=r\mathrm{e}^{j\theta}$ then
            \[
                z^*=\mathrm{e}^{-j\theta} = \cos{\theta}-j\sin{\theta}.
            \]
    \end{enumerate}
}
