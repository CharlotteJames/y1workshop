\multiproblem{1}{Let's begin with some straightforward proofs to get you warmed up. Prove the following statements hold $\forall~n\in\mathbb{N}$ by induction:
\begin{enumerate}
\item \[ P(n):=\quad \sum_{i=1}^{n} i = \frac{1}{2} n (n+1) \] (this is the example \eqref{eq:ex1} given above).
\item \[ P(n):=\quad n^{2}=(n-1)^{2}+2n-1. \]
\item \[ P(n):=\quad \sum_{i=1}^{n}(2i-1)=n^2. \]
\item \[ P(n):=\quad \sum_{i=0}^{n} 2^i = 2^{n+1}-1. \]
\item \[ P(n):=\quad \sum_{i=1}^{n} 3i-2=\frac{n}{2}(3n-1) .\]
\item \[ P(n):=\quad \sum_{i=1}^{n} i^{2} = \frac{1}{6}n(n+1)(2n+1). \]
\item \[ P(n):=\quad \sum_{i=1}^{n} \frac{1}{i(i+1)} = \frac{n}{n+1}. \]
\item \[ P(n):=\quad \sum_{i=1}^{2n} i^{2} = \frac{1}{3}n(2n+1)(4n+1). \]
\item \[ P(n):=\quad \sum_{i=1}^{n} i^{3} = \frac{1}{4}n^{2}(n+1)^{2}. \]
\item \[ P(n):=\quad \sum_{i=0}^{n}r^{i}=\frac{1-r^{n+1}}{1-r}, \quad r\neq 1.\]
\end{enumerate}
}

\multiproblem{2}{Now let's consider some slightly more tricky ones. By induction:
\begin{enumerate}
 \item Prove that $(1+p)^{n} \geq 1+np$, where $p>-1$, $n\in \mathbb{N}$.
 \item Prove that $n^{3}+2n$ is divisible by 3, for $n \in \mathbb{N}$..
 \item Prove that $6^{n}-1$ is divisible by 5, for $n \in \mathbb{N}$.
 \item Prove that $8^{2n}-1$ is divisible by 63, for $n \in \mathbb{N}$.
 \item Prove that the number $2^{4n-1}$ ends with an 8, for $n \in \mathbb{N}$.
 \item Prove that $n!>2^{n}$ for $n\in\mathbb{N}$ and $n\geq 4$.
 \item Prove that $n!>3^{n}$ for $n\in\mathbb{N}$ and $n\geq 7$.
 \item Prove that $\frac{\mathrm{d}^{n}}{\mathrm{d}x^{n}}x^{n}=n!$, for $n \in \mathbb{N}$.
 \item Prove that $\int_{0}^{\infty}x^{n}\mathrm{e}^{-x}\mathrm{d}x=n!$, for $n \in \mathbb{N}$. (Hint: Use integration by parts).
 \item
A natural number $n$ has a prime decomposition if it can be written as 
\[
 n=\prod_{i=1}^{N} p_{i}^{\alpha_{i}}= p_{1}^{\alpha_{1}}p_{2}^{\alpha_{2}}\dots p_{N}^{\alpha_{N}},\quad \text{where} \quad \alpha_{i} \in \mathbb{N} 
\]
and $p_{i}$ are distinct prime numbers.
 Prove that every natural number greater than or equal to 2 has a prime decomposition. (Hint: Use strong induction).
\end{enumerate}
}

\multiproblem{3}{Let ${\bf A}$ be an $n\times n$ square matrix.
\begin{enumerate}
\item Suppose that $\lambda$ is an eigenvalue of the matrix ${\bf A}$, and ${\bf v}$ is the corresponding eigenvector. Show that $\lambda^{2}$ is an eigenvalue of ${\bf A}^{2}$.
\item Prove by induction that $\lambda^{n}$ is an eigenvalue of ${\bf A}^{n}$, for $n \in \mathbb{N}$.
\item
 Now suppose we now define a square upper triangular matrix of dimension $n$ as 
 \[
 {\bf A}_{n}=
  \begin{pmatrix}
   a_{11} & a_{12} & \hdots & a_{1n} \\
   0 & a_{22} & \hdots & a_{2n} \\
   \vdots & 0 & \ddots & \vdots \\
   0 & 0 & \hdots & a_{nn} 
  \end{pmatrix}.
 \]
 Prove by induction that $\text{det}({\bf A}_{n})=\prod_{i=1}^{n}a_{ii}$ for $n\geq 2$, $N\in \mathbb{N}$.
\end{enumerate}
}

\multiproblem{4}{
The Fibonacci sequence, $\left\{ x_{n}\right\}$ is defined by the relations $x_{1}=x_{2}=1$ and $x_{n}=x_{n-1}+x_{n-2}$ for $n\geq 3$. 
\begin{enumerate}
\item Find the first 12 terms in the Fibonacci sequence.
\item Now let us prove by induction that 
\begin{equation}\label{fib}
 x_{n}=\frac{1}{\sqrt{5}}\left( \left( \frac{1+\sqrt{5}}{2} \right)^{n}-\left( \frac{1-\sqrt{5}}{2} \right)^{n} \right).
\end{equation}
First demonstrate that the statement holds for $n=1$ {\em and} $n=2$. This is the Base Step.
\item Now for the Inductive Step. By assuming that $P(n)$ and $P(n+1)$ hold, show that $P(n+2)$ follows. This completes the proof by induction.
 \end{enumerate}
 }

\multiproblem{5}{The power set of $S$, denoted $\mathcal{P}(S)$, is the set of every possible subset of $S$, including the empty set and $S$ itself. The cardinality of $S$, denoted $|S|$, is the number of elements in the set $S$. Now let $S_{n}=\left\{1,2,\dots,n\right\}$.
\begin{enumerate}
\item We know that $\mathcal{P}(S_{n+1})$ can be written in terms of $\mathcal{P}(S_{n})$ as follows:
\begin{equation}\label{ps}
 \mathcal{P}(S_{n+1})=\mathcal{P}(S_{n})\cup\left\{S\cup{n+1}:S\in\mathcal{P}(S_{n}) \right\}.
\end{equation}
Show that this statement is true for $n=1$ and $n=2$.
\item For sets $A$ and $B$ we know that 
\begin{equation}\label{sets}
 |A \cup B|=|A|+|B|-|A\cap B|.
\end{equation}
Using the results \eqref{ps} and \eqref{sets}, prove by induction that $|\mathcal{P}(S_{n})|=2^{n}$ for all $n\in\mathbb{N}$.
\end{enumerate}
}

\multiproblem{6}{The Binomial Theorem states that
\begin{equation}\label{bin}
 (a+b)^{n}=\sum_{r=0}^{n} \binom{n}{r} a^{r} b^{n-r}
\end{equation}
where $a,b \in \mathbb{R}$ and $n\in \mathbb{N}$. The binomial coefficient is defined
\[
 \binom{n}{r}=\frac{n!}{r!(n-r)!}.
\]
\begin{enumerate}
 \item Show that
 \[
  \binom{n}{r-1}+\binom{n}{r}=\binom{n+1}{r}.
 \]
 \item Establish that the Binomial Theorem holds where $n=1$ (the Base Step).
 \item Show that if we assume \eqref{bin} (i.e. that the Binomial Theorem holds for $n$) then
 \[
  (a+b)^{n+1}=(a+b)(a+b)^{n}=\sum_{r=1}^{n+1}\binom{n}{r-1}a^{r}b^{n+1-r}+\sum_{r=0}^{n}\binom{n}{r}a^{r}b^{n+1-r}.
 \]
(Hint: the trick is to make a substitution of $s=r+1$ in the first summation term after expanding).
\item Prove the Binomial Theorem by induction (Hint: You have already demonstrated the Base Step. To complete the Inductive Step, work from the result you have found in (c). It will be helpful to use the result regarding binomial coefficients that you found in (a)).
\end{enumerate}
}