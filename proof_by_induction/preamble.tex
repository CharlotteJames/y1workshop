Proof by induction is a form of direct proof that is very widely used. It is a mathematical proof technique that allows us to prove that a given statement holds for all the natural numbers greater than some initial value.

More precisely suppose we have a statement, denoted $P(n)$, that we wish to prove is true for all values of $n \in \mathbb{N}$. Here $\mathbb{N}=\left\{1,2,3\dots\right\}$ denotes the set of natural numbers, using the definition that excludes zero. Now for instance we might want to show
\begin{equation} \label{eq:ex1}
 P(n):=\quad 1+2+3+\dots+n=\frac{n(n+1)}{2}\quad\forall ~n\in\mathbb{N}.
\end{equation}
The technique of proof by induction allows us to prove this. The proof is comprised of the following two steps:
\begin{enumerate}
 \item {\bf Base Step:} Demonstrate that $P(n)$ is true for the for the initial value of $n \in \mathbb{N}$. 
 \item {\bf Inductive Step:} Show that $P(n) \implies P(n+1)$ for some $n \in \mathbb{N}$.
\end{enumerate}

Note that the value of $n$ chosen for the Base Step will depend on which values of $n$ we are trying to prove the statement $P(n)$ is true for. Often we will be proving that the statement $P(n)$ is true for all the natural numbers, in which case the initial value will be $n=1$. However, sometimes we may wish to prove a statement holds true for all natural numbers greater than a certain value that is larger than 1. In this case, it is still possible to use induction but our initial value used in the Base Step will be different.

For the Inductive Step, we are showing that the {\em implication} $P(n) \implies P(n+1)$ is true. To do this, we need to show $P(n+1)$ cannot be false when $P(n)$ is true. Therefore in this step we first assume $P(n)$ is true. Here, $P(n)$ is the {\em inductive hypothesis} or assumption. Then to complete the Inductive Step we use $P(n)$ to show $P(n+1)$. This demonstrates that if $P(n)$ is assumed to be true, then $P(n+1)$ is also true. 

With both these steps, the proof is complete. The Base Step shows that $P(1)$ is true. From the Inductive Step we know that $P(n)\implies P(n+1)$. Therefore $P(1)\implies P(2)$, so $P(2)$ is also true. Now from the inductive step again, $P(2)\implies P(3)$ and so $P(3)$ is true. This process can be repeated ad infinitum. So $P(n)$ is true for all $n \in \mathbb{N}$ by induction.

Note that in the Inductive Step we are not assuming that $P(n)$ is true {\em for all} $n\in \mathbb{N}$, but just for a given $n$. There is a variant of induction, called {\em strong induction}, where for the Inductive Step we assume that $P(n)$ holds for all values of $n$ up to and including $n$.