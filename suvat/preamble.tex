The SUVAT equations describe the motion of a particle under constant acceleration in a line. The assumption of constant acceleration simplifies the description of the motion from a system of differential equations to 5 algebraic expressions in: $t$ the time; $s$ the displacement after time $t$; $u$ the initial velocity at time $t=0$; $v$ the final velocity at time $t$ and $a$ the acceleration. 

The SUVAT equations are:
\begin{itemize}
	\item S(1): $v = u+at$ 
    \item S(2): $s=ut+\frac{1}{2}at^2$
    \item S(3): $s = \frac{1}{2}(u+v)t$
    \item S(4): $v^2 = u^2 + 2as$
    \item S(5): $s=vt-\frac{1}{2}at^2$
\end{itemize}
The derivation is worked through in one of the questions below. Note how each equation contains four of the five variables, this means we only ever need to know three of the quantities in order to calculate the final two.

The generalisation of the SUVAT equations to include motion in the plane is simple: $s$, $u$, $v$ and $a$ become vectors, with a horizontal and vertical component. S(4) is the only equation we need to consider more carefully, as it has some of these quantities multiplying each other and produces a scalar solution.

From Newton's Second Law of Motion ($F=ma$), we can see that a constant acceleration implies a constant force acting on the object (taking $m$ constant). This also allows us to calculate quantities such as work done on the object, and the kinetic energy of the object at a given time. 