This worksheet builds on the earlier worksheet on probability, introducing the idea of random variables, mean and variance and some common probability distributions.\\

\subsubsection*{Random variables}

Let $S$ be the sample space for a random trial. Now define a function $\phi: S\to \mathbb{R}$. Then $\phi$ acting on $S$ defines a random variable, written $X$.\\
\newline
{\bf Example}\\
Consider the random trial of tossing a fair coin twice. Then $S = \left\{HH, HT, TH, TT \right\}$. We could define our random variable $X$ to be the number of heads obtained, and so we would have $\phi(HH)=2$, $\phi(HT)=1$, $\phi(TH)=1$ and $\phi(TT)=0$.

Random variables can either be discrete or continuous.
\begin{itemize}
 \item {\em Discrete} random variables are defined on the sample space of a random trial with a set of discrete outcomes (e.g. throwing a die or tossing a coin).
 \item {\em Continuous} random variables are defined on the sample space of a random trial where the outcome can take any value from a range (e.g. the heights of people).
\end{itemize}

Some of the key concepts for discrete and continuous random variables are given in the table below.\\

\begin{center}
    \begin{tabular}{|p{7cm} | p{7cm}|}
    \hline
    {\bf Discrete} & {\bf Continuous} \\ \hline
    Probability function:  & Probability density function (pdf):  \\
    $P_{X}(x):=P(X=x)$ & $P(x_{1}\leq x \leq x_{2})=\int_{x_{1}}^{x_{2}}f_{X}(x)\:\mathrm{d}x$ \\ \hline
    Cumulative distribution function (cdf): & Cumulative distribution function (cdf): \\
    $F_{X}(x):=P(X\leq x)=\sum_{\tilde{x} \leq x} P_{X}(\tilde{x})$ & $F_{X}(x):=P(X\leq x)=\int_{-\infty}^{x}f_{X}(\tilde{x})\mathrm{d}\tilde{x}$ \\ \hline
    Expected value (mean): & Expected value (mean): \\ 
    $\mu=E(X):=\sum_{x} x P_{X}(x)$ & $\mu=E(X)=\int_{-\infty}^{\infty}x f_{X}(x)\: \mathrm{d}x$ \\
    \hline
    \end{tabular}
\end{center}

The {\em variance} of any random variable is defined in terms of the expected value as
\begin{equation*}
\text{Var}(X)=\sigma^{2}:=E((X-\mu)^{2})=E(X^{2})-\mu^{2}.
\end{equation*}
This looks like a lot of maths! Let's look at a simple example to try to understand what is going on.\\
\newline
{\bf Example}\\
Consider the random trial of tossing a fair coin three times in succession. Suppose we are just interested in how many times we get tails. Therefore we define our random variable $X$ to be the number of tails. So we have $\phi(HHH)=0$, $\phi(HHT)=1$ etc. as in our earlier example. The number of tails $X$ can either be 0, 1, 2 or 3. The probability function just gives the probability of our random variable $X$ taking the value $x$. For instance $P_{X}(0)=\frac{1}{8}$. The probability function is shown in the table below. 

\begin{center}
\begin{tabular}{ |c|c|c|c|c| } 
 \hline
 $x$ & 0 & 1 & 2 & 3 \\ 
\hline $P_{X} (x)$& $\frac{1}{8}$ & $\frac{3}{8}$ & $\frac{3}{8}$ & $\frac{1}{8}$ \\
 \hline
\end{tabular}
\end{center}

The cdf gives us the probability that our random variable $X$ takes a value less than or equal to x. For instance
\begin{equation*}
F_{X}(2)=P(X\leq 2)=\sum_{\tilde{x} \leq 2} P_{X}(\tilde{x})=P_{X}(0)+P_{X}(1)+P_X(2)=\frac{1}{8}+\frac{3}{8}+\frac{3}{8}=\frac{7}{8}.
\end{equation*}
This is just saying that there is a $\frac{7}{8}$ probability for the number of tails $X\leq 2$. The full cdf is given in the table below.

\begin{center}
\begin{tabular}{ |c|c|c|c|c| } 
 \hline
 $x$ & 0 & 1 & 2 & 3 \\ 
\hline $F_{X} ( x)$& $\frac{1}{8}$ & $\frac{4}{8}$ & $\frac{7}{8}$ & $1$ \\
 \hline
\end{tabular}
\end{center}

The expected value $E(X)$ can be found by summing up $x P_{X}(x)$ for all values our random variable can take. So for our example:
\begin{equation*}
 \mu=E(X)=\sum_{x} x P_{X}(x)=0 \frac{1}{8}+1 \frac{3}{8} + 2 \frac{3}{8} + 3\frac{1}{8}=1.5.
\end{equation*}
This tells us that the expected number of heads from tossing a fair coin 3
times is 1.5.

Note that on any perticular outcome we will always get an integer number of
heads: the expected value of a random variable does not need to correspond to
any possible outcome of the random trial. It represents the mean value of the
random variable if we were to repeat the trial a large number of times.

\subsubsection*{Distributions}
Depending on what you trying to model, there are a number of common distributions that can capture what is going on. A summary of the distributions you have covered so far and their probability functions/pdfs is given below.

\begin{center}
    \begin{tabular}{|p{2.5cm} | p{2.5cm}| p{4cm} | p{5cm}|}
    \hline
    {\bf Distribution} & {\bf Type} & {\bf Parameters} & {\bf Probability function/pdf} \\ \hline
    Bernoulli & Discrete & Prob. success $p$ & $P(0)=1-p$, $P(1)=p$ \\ \hline
    Geometric & Discrete & Prob. success $p$ & $P(k)=p(1-p)^{k-1}$ \\ \hline
    Binomial & Discrete & Prob. success $p$, no. of trials $n$ & $P(k)=\binom{n}{k} p^{k} (1-p)^{n-k}$ \\ \hline
    Poisson & Discrete & Rate $\lambda$ & $P(k)=\frac{\lambda^{k}}{k!}\exp(-\lambda)$ \\ \hline
    Exponential & Continuous & $\lambda$ & $f_{X}(x)=\lambda\exp (-\lambda x)$ if $x\geq 0$, else 0 \\ \hline
    Normal & Continuous & $\mu$, $\sigma$ & $f_{X}(x)=\frac{1}{\sqrt{2 \pi} \sigma}\exp \left( - \frac{(x-\mu)^{2}}{2 \sigma^{2}} \right)$ \\
    \hline
    \end{tabular}
\end{center}

Now for some questions...\\
