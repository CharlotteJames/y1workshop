%
%

\multiproblem{linear}{

    We now consider the equation
    \begin{equation}
        a\,x = b
        \label{eq:axb}
    \end{equation}
    where $a$ and $b$ are known and $x$ is an unknown quantity. We begin by
    knowing that x is from some set say $S$. Equation~\eqref{eq:axb} implies
    that $x$ in fact belongs to a set $T$ which is a subset of $S$
    ($T\subseteq S$).  Our objective is to find the set $T$ comprising all
    solutions of equation~\eqref{eq:axb} for $x\in S$.

    \begin{enumerate}
        \item Let's first consider some examples. What is the solution set
            for $x$ in the following cases?
            \begin{enumerate}
                \item $2\,x = 8$ and $x\in\mathbb{R}$
                \item $0\,x = 8$ and $x\in\mathbb{R}$
                \item $0\,x = 0$ and $x\in\mathbb{R}$
            \end{enumerate}
        \item More generally given $a,b,x\in\mathbb{R}$ how does the solution
            set for~\eqref{eq:axb} depend on $a$ and $b$?
        \item What difference does it make if $a,b,x\in\mathbb{Q}$ or
            $a,b,x\in\mathbb{C}$?
        \item What if $a,b,x\in\mathbb{Z}$?
        \item Consider the same equation in modular arithmetic
            \[
                a\,x = b \Mod{n}
            \]
            with $a,b,x\in\mathbb{Z}$ and $n\in\mathbb{N}$.
            \begin{enumerate}
                \item If $x = 1 \Mod{3}$ what is the solution set for $x$?
                \item If $a\,x = b \Mod{p}$ with $p$ prime what are the sets of
                    solutions for different values of $a$ and $b$?
                \item What could happen to the solution set if $p$ is not
                    prime? (consider e.g. $3\,x=3\Mod{6}$ or $3\,x=2\Mod{6})$.
            \end{enumerate}
    \end{enumerate}
}

\multiproblem{quadratic}{
    It's common for mathematical problems to be posed in such a way that the
    number of solutions (the cardinality of the solution set) will be either
    $0$, $1$ or $\infty$. When the number of solutions is $0$ we say that a
    solution does not \emph{exist}. If a solution exists but there is more
    than one we say that we do not have a \emph{unique} solution. In many
    areas there will be conditions guaranteeing existence and uniqueness so
    that we can know when there will be exactly one solution.

    The most common exception to the $1,0,\infty$ pattern is when dealing
    with polynomial equations. Here we consider the quadratic equation
    \begin{equation}
        a\,x^2 + b\,x + c = 0.
        \label{eq:quad}
    \end{equation}
    Note that if $a=0$ then this is equivalent to~\eqref{eq:axb}. Often we say
    that by definition a quadratic equation must have $a\neq 0$. In the
    questions below carefully consider how the assumption $a\neq 0$ would
    affect the set of solutions.
    \begin{enumerate}
        \item Given $a,b,c,x\in\mathbb{R}$ what will be the set of solutions
            to~\eqref{eq:quad}? Carefully consider all cases (e.g. $a=0$ or
            perhaps $a=b=0$ but $c\neq 0$ etc.). Under what conditions can we
            guarantee existence or uniqueness of solutions?
        \item What happens to the sets of solutions to~\eqref{eq:quad} if
            $a,b,c,x\in\mathbb{C}$? Under what conditions on $a,b,c$ can we
            guarantee the existence of solutions in this case? (This
            property of complex numbers is called \emph{algebraic closure}.)
        \item Now consider the case that $a,b,c,x\in\mathbb{Z}$. Under what
            conditions do we have existence/uniqueness of solutions?
        \item Given $a,b,c,x\in\mathbb{Z}$ show that if a solution
            to~\eqref{eq:quad} exists then it divides $c$.

            This principle naturally extends to higher-order polynomials so
            show that there are no integer solutions to
            \begin{equation}
                2x^5 - x^4 + 4x^3 - 2x^2 + 2x - 1 = 0
                \label{eq:quintic}
            \end{equation}
        \item Given $a,b,c,x\in\mathbb{Q}$ we can write $a=\frac{n_a}{n_d}$
            etc. and then multiply through by the square of all of the
            denominators to obtain an equation in the same form
            as~\eqref{eq:quad} but with $a,b,c\in\mathbb{Z}$ and
            $x\in\mathbb{Q}$ (prove/verify this).

            Suppose that $a,b,c\in\mathbb{Z}$ and $x\in\mathbb{Q}$ and that
            there exists a rational solution $x=\frac{n}{d}$ with
            $d,n\in\mathbb{Z}$. Show that $n$ divides $c$ and $d$ divides $a$.

            How would this generalise to higher order polynomials? Hence find
            the set of all rational solutions to~\eqref{eq:quintic}.
    \end{enumerate}
}

\multiproblem{physics}{
    \begin{enumerate}
        \item Consider this system:\newline
            \includegraphics{mass_horizontal.pdf}
    \end{enumerate}<++>
}
