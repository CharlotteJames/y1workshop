In this exercise sheet we will consider solutions of equations/inequalities
and systems of equations/inequalities. Given an unknown quantity $x$ we would
often think of a solution as something like $x=\cdots$. Instead here we will
think of a set $S$ of solutions such that $x\in S$.

It is important in any mathematical problem to consider carefully which sets
the solutions must be drawn from. In many physical problems it is clearly
required that an unknown quantity must be e.g. an integer or a real number.
Often (but not always) the distinction between real numbers and rational
numbers is unimportant. However the distinction between real and (non-real)
complex numbers is often important -- complex solutions are often considered
to represent the absence of real solutions. Likewise if $x$ must be an integer
finding that $x=\frac{1}{2}$ implies that no solution for $x$ exists.

Let's consider an example to illustrate what we mean by a set of solutions.
Given
\[
    x^2 = 2
\]
with $x\in\mathbb{R}$ what is the solution set for $x$? We said that
$x\in\mathbb{R}$ which means that $x$ must be a real number. However not all
real numbers satisfy the equation. In fact there are only two real values for
$x$ which can satisfy the equation: $+\sqrt{2}$ and $-\sqrt{2}$. We would
often write this as
\[
    x = \pm\sqrt{2}
\]
however here we would like to explicitly state this in terms of sets
\[
    x\in\left\{\sqrt{2},-\sqrt{2}\right\}.
\]

Had our problem statement required that $x\in\mathbb{Z}$ ($x$ must be an
integer) then there would be no solutions. In that case the set of solutions
would be the $\emptyset$ (the empty set).
